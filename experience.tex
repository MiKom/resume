\begin{rubric}{\inEnglish{Experience}\inPolish{Doświadczenie}}
\noentry{2011.09 -- 2012.01}
\subrubric{\inEnglish{Employment}\inPolish{Zatrudnienie}}
\entry*[2016.02 -- \ldots]
	\inEnglish{\textbf{Setapp sp. z o.o.} --- Scrum Master in multiple web and mobile products.
		Involvement in shaping the software development processes in the
		company. Helping sales team in evaluation of incoming leads. Member of the business
		\href{https://www.scrum.org/resources/agility-guide-evidence-based-change}{Evidece-Based Change}
		team that conducts a set of improvements to the operations of the company.
	}
	\inPolish{\textbf{Setapp sp. z o.o.} --- Scrum Master w produktach webowych i mobilnych.
		Udział w kształtowaniu procesów rozwoju oprogramowania w firmie.
		Wsparcie działu sprzedażowego w ocenie potencjalnych projektów. Udział w zespole zmiany biznesowej
		opartym o metodologię \href{https://www.scrum.org/resources/agility-guide-evidence-based-change}{Evidece-Based Change}.
	}

\entry*[2015.08 -- 2016.02]
	\inEnglish{\textbf{Setapp sp. z o.o.} --- Scrum master/developer.
		Development of VR projects:
	\begin{itemize}
		\item \textbf{Escape Velocity} one of the first UE4 experiences in
			Oculus Mobile Store. Duties:~developer and Scrum Master.
		\item \textbf{Neverout} critically acclaimed puzzle game for Gear VR.
			Duties:~Scrum Master, testing, release management.
	\end{itemize}
	}
	\inPolish{\textbf{Setapp sp. z o.o.} --- Scrum master/developer.
		Rozwój projektów VR:
	\begin{itemize}
		\item \textbf{Escape Velocity} jedna z pierwszych aplikacji na
			silniku UE4 w sklepie Oculus Mobile. Role:~developer,
			Scrum Master.
		\item \textbf{Neverout} dobrze przyjęta gra logiczna na Gear VR.
			Role:~Scrum Master, testowanie, zarządzanie wydaniami.
	\end{itemize}
	}
\entry*[2013.11 -- 2015.08]
	\inEnglish{\textbf{Dark Stork Studios} --- team leader/developer.

	Development of video game project using Unreal Engine 4 (UE4) and technical project management.

	Upon joining the project and quickly gaining knowledge of UE4, I was one of the initiators
	and implementors of structured development process based on Scrum and adapted for game development.
	
	When the production started and the process was in place my duties included:
	\begin{itemize}
		\item Management of process for 5-6 game developers in Scrum Master role.
		\item Gathering and managing requirements from project stakeholders. Initial estimation,
		sprint and release planning.
		\item Implementation and management of continuous integration and deployment.
		\item Deployment and administration of Atlassian tools stack.
		\item Coordinating communication with external software vendors.
		\item Day--to--day development and design of core product components.
	\end{itemize}
	}
	\inPolish{\textbf{Dark Stork Studios} --- team leader/developer.

	Programowanie gry komputerowej na silniku Unreal Engine 4 (UE4) i techniczne zarządzanie projektem.

	Po dołączeniu do projektu szybkim zdobyciu wiedy na temat UE4 byłem jednym z inicjatorów wykonaców
	wdrożenia ustrukturyzowanego procesu wytwarzania produktu przy użyciu frameworka Scrum dostosowanego
	do branży gier komputerowych.

	W trakcie finalnej produkcji, do moich obowiązków należały:
	\begin{itemize}
		\item Zarządzanie procesem w grupie 5-6 programistów gry w roli Scrum Mastera.
		\item Zbieranie i zarządzanie wymaganiami od interesariuszy projektu. Wstępna ocena nakładu
		pracy, planowanie wydań i sprintów.
		\item Wdrożenie i zarządzanie systemami CI i CD.
		\item Wdrożenie i administracja zestawem narzędzi deweloperskich firmy Atlassian.
		\item Koordynowanie współpracy z zewnętrznymi dostawcami technologii.
		\item Codzienny rozwój i projektowanie kluczowych komponentów produktu.
	\end{itemize}
	}
\entry*[2012.08 -- 2013.10]
	\inEnglish{\textbf{Samsung Electronics} --- junior software engineer.

	Debugging and development of applications and kernel drivers for Linux--based embedded devices (Set-top-boxes).

	Development of STB diagnostics application in Qt4 with QML UI.

	Performing analysis of opens source software usage to prevent licensing problems.

	Preparing documents with formal requirements for STB diagnostics application}
\inPolish{\textbf{Samsung Electronics} --- junior software engineer.

	Debugowanie i rozwój aplikacji i sterowników kernelowych w Linuksowych urządzeniach wbudowanych (Set-top-box).

	Rozwój aplikacji diagnostycznej na STB z wykorzystaniem Qt4 i QML.

	Przeprowadzanie audytu oprogramowania open source w celu uniknięcia problemów licencyjnych.

	Tworzenie dokumentów z formalnymi wymaganiami dla aplikacji diagnostycznej na STB.}
\entry*[2010.09 -- 2012.08]
	\inEnglish{\textbf{Adam Mickiewicz University, Faculty of Mathematics and Computer Science} --- system administrator.
	I was a member of a team of five people that maintained and expanded faculty's IT
	infrastructure. We were maintaining networking equipment and its configuration, 
	about 200 dual-boot Windows/GNU Linux workstations, about 30 Windows/Linux/VMWare
	ESXi servers and numerous services running on those servers. We have also 
	been providing IT troubleshooting for faculty employees and students.}
	\inPolish{\textbf{Uniwersytet im. Adama Mickiewicza, Wydział Matematyki i Informatyki} --- administrator.
	Byłem członkiem pięcioosobowego zespołu, który utrzymywał i rowijał 
	wydziałową infrastrukturę informatyczną. Obsługiwaliśmy i konfigurowaliśmy m.in. urządzenia
	sieciowe wydziału, około 200 stacji roboczych z systemami Windows i GNU/Linux Ubuntu,
	około 30 serwerów z systemami Windows, GNU Linux i VMWare. Świadczyliśmy również wsparcie
	techniczne dla pracowników wydziału i studentów}
%\entry*[2009.09]
%	\inEnglish{\textbf{Beyond.pl Sp. z o.o.} internship (3 weeks)}
%	\inPolish{\textbf{Beyond.pl Sp. z o.o.} --- praktyki studenckie (3 tyg.)}

%\subrubric{\inEnglish{Highlighted projects}\inPolish{Wybrane projekty}}
%\entry*[2011.09 -- 2012.04]
%	\inEnglish{\textbf{CLIPT - OpenCL Image Processing Toolkit} --- I'm
%	co-author and current maintainer of small \GTK application that
%	implements various image processing algorithms on GPU through OpenCL.}
%	\inPolish{\textbf{CLIPT - OpenCL Image Processing Toolkit} --- jestem
%	współautorem i aktualnym opiekunem niewielkiej aplikacji w toolkicie
%	\GTK, w której zaimplementowane zostały różne algorytmy przetwarzania
%	obrazów na kartach graficznych przy użyciu OpenCL.}
%	\entry* \url{https://github.com/MiKom/clipt}
%\entry*[2009 -- 2012]
%	\inEnglish{\textbf{Vorticity engine} --- ongoing development of research game engine. I co-authored IO subsystem, ported the engine
%	from Visual Studio to CMake build system, and ported platform-dependent bits to X11. It is currently used as a renderer
%	for my masters thesis project.}
%	\inPolish{\textbf{Silnik 3D \emph{Vorticity}} --- opieka nad projektem eksperymentalnego
%	silnika gier 3D. Jestem współautorem podsystemu IO silnika. Ponadto przeniosłem silnik
%	z systemu budowania Visual Studio na CMake i stworzyłem port części silnika
%	zależnych od platformy dla środowiska XWindows, co umożliwiło uruchomienie go
%	na systemie GNU/Linux. Aktualnie silnik jest wykorzystywany w mojej pracy
%	magisterskiej.}
%	\entry* \url{https://github.com/MiKom/vorticity}
%\entry*[2010]
%	\inEnglish{\textbf{Arsen} --- augmented reality game based on early version
%	of Vorticity. Created for \emph{Team Project} course at university. It's a multiplayer
%	3D game controlled by Augmented Reality (AR) interface. I implemented most of
%	the gameplay mechanics, networking code and AR interface.}
%	\inPolish{\textbf{Arsen} --- gra komputerowa z wykorzystaniem rzeczywistości
%	rozszerzonej. Stworzona w ramach przedmiotu \emph{Projekt Zespołowy} podczas
%	studiów na uniwersytecie. Zaimplementowałem większość mechaniki rozgrywki,
%	kod sieciowy i interfejs rozszerzonej rzeczywistości}
%\entry*[2009]
%	\inEnglish{Junior Java developer in Poznań team of European Space Agency sposored GENSO project}
%	\inPolish{Młodszy programista Java w Poznańskiej sekcji projektu Europejskiej Agencji Kosmicznej GENSO}
%	\entry* \url{http://www.genso.org}
\end{rubric}

