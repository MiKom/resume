\begin{rubric}{\inEnglish{Experience}\inPolish{Doświadczenie}}
\noentry{2011.09 -- 2012.01}
\subrubric{\inEnglish{Employment}\inPolish{Zatrudnienie}}
\entry*[2009.09]
	\inEnglish{\textbf{Beyond.pl Sp. z o.o.} internship (3 weeks)}
	\inPolish{\textbf{Beyond.pl Sp. z o.o.} --- praktyki studenckie (3 tyg.)}
\entry*[2010.09 -- \ldots]
	\inEnglish{\textbf{Adam Mickiewicz University, Faculty of Mathematics and Computer Science} system administrator.
	I am a member of a team of five people that maintains and expands faculty's IT
	infrastructure. We maintain networking equipment and it's configuration, 
	about 200 dual-boot Windows/GNU Linux workstations, about 30 Windows/Linux/VMWare
	ESXi servers and numerous services running on those servers. We also 
	provide IT troubleshooting for faculty employees and students.}
	\inPolish{\textbf{Uniwersytet im. Adama Mickiewicza, Wydział Matematyki i Informatyki} --- administrator.
	Jestem członkiem pięcioosobowego zespołu, który utrzymuje i rozwija
	wydziałową infrastrukturę informatyczną. Obsługujemy i konfigurujemy m.in. urządzenia
	sieciowe wydziału, około 200 stacji roboczych z systemami Windows i GNU/Linux Ubuntu,
	około 30 serwerów z systemami Windows, GNU Linux i VMWare. Świadczymy również wsparcie
	techniczne dla pracowników wydziału i studentów}

\subrubric{\inEnglish{Highlighted projects}\inPolish{Wybrane projekty}}
\entry*[2009]
	\inEnglish{Junior Java developer in Poznań team of European Space Agency sposored GENSO project}
	\inPolish{Młodszy programista Java w Poznańskiej sekcji projektu Europejskiej Agencji Kosmicznej GENSO}
	\entry* \url{http://www.genso.org}
\entry*[2010]
	\inEnglish{\textbf{Arsen} --- augmented reality game based on early version
	of Vorticity. Created for \emph{Team Project} class at university. It's a multiplayer
	3D game controlled by Augmented Reality (AR) interface. I implemented most of
	the gameplay mechanics, networking code and AR interface.}
	\inPolish{\textbf{Arsen} --- gra komputerowa z wykorzystaniem rzeczywistości
	rozszerzonej. Stworzona w ramach przedmiotu \emph{Projekt Zespołowy} podczas
	studiów na uniwersytecie. Zaimplementowałem większość mechaniki rozgrywki,
	kod sieciowy i interfejs rozszerzonej rzeczywistości}
\entry*[2009 -- \ldots]
	\inEnglish{\textbf{Vorticity engine} --- ongoing development of research game engine. I co-authored IO subsystem, ported the engine 
	from Visual Studio to CMake build system, and ported platform-dependent bits to X11. It is currently used as a renderer
	for my masters thesis project.}
	\inPolish{\textbf{Silnik 3D \emph{Vorticity}} --- opieka nad projektem eksperymentalnego
	silnika gier 3D. Jestem współautorem podsystemu IO silnika. Ponadto przeniosłem silnik
	z systemu budowania Visual Studio na CMake i stworzyłem port części silnika
	zależnych od platformy dla środowiska XWindows, co umożliwiło uruchomienie go
	na systemie GNU/Linux. Aktualnie silnik jest wykorzystywany w mojej pracy
	magisterskiej.}
	\entry* \url{https://github.com/MiKom/vorticity}
\entry*[2011.09 -- \ldots]
	\inEnglish{\textbf{CLIPT - OpenCL Image Processing Toolkit} --- I'm
	co-author and current maintainer of small \GTK application that
	implements various image processing algorithms on GPU through OpenCL.}
	\inPolish{\textbf{CLIPT - OpenCL Image Processing Toolkit} --- jestem
	współautorem i aktualnym opiekunem niewielkiej aplikacji w toolkicie
	\GTK, w której zaimplementowane zostały różne algorytmy przetwarzania
	obrazów na kartach graficznych przy użyciu OpenCL.} 
	\entry* \url{https://github.com/MiKom/clipt}
\end{rubric}

